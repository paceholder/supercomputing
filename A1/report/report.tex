\documentclass[utf8,12pt]{report}
\usepackage{pscyr}
\usepackage[T1]{fontenc}
\usepackage[english]{babel}
\usepackage{amssymb,amsfonts,amsmath,mathtext}
\usepackage{cite,enumerate,float,indentfirst}
\usepackage[utf8]{inputenc}
\usepackage{ragged2e}
\justifying
\usepackage{epstopdf}
\usepackage{listings}
\usepackage[pdftex]{graphicx}
\usepackage{color}
\usepackage{lmodern}
\input{rgb}


\lstset{% general command to set parameter(s)
    basicstyle=\ttfamily\small, % \small, % print whole listing small
    commentstyle=\color{white}, % white comments
    stringstyle=\ttfamily, % typewriter type for strings
    showstringspaces=false} % no special

\renewcommand{\raggedright}{\leftskip=0pt \rightskip=0pt plus 0cm}

\setcounter{tocdepth}{2}

\graphicspath{{images/}}

\title{lalalal}

\begin{document}

\title{Programming of supercomputers. Assignmeng sheet 1}
\author{Dmitry Pinaev}
\date{\today}
\maketitle



\section{System configuration}

We have a system of 16 identical processors with following charachteristics obtained with the command less /proc/cpuinfo

\begin{table}
	\begin{tabular}{|l|l|}
	\hline
		vendor\_id & AuthenticAMD \\ 
		cpu family & 16 \\
		model      & 9 \\
		model name & AMD Opteron(tm) Processor 6128 HE \\
		stepping   & 1 \\
		cpu MHz    & 2000.000 \\
		cache size & 512 KB \\
		physical id& 0 \\
		siblings   & 8 \\
		core id    & 0 \\
		cpu cores  & 8 \\
		apicid     & 16 \\
		initial apicid  & 0 \\
		fpu             & yes \\
		fpu\_exception  & yes \\
		cpuid level     & 5 \\
		wp              & yes \\
		flags           & fpu vme de pse tsc msr pae mce cx8 apic \\
		                & sep mtrr pge mca cmov pat pse36 clflush mmx \\
		                & fxsr sse sse2 ht syscall nx mmxext fxsr\_opt \\
		                & pdpe1gb rdtscp lm 3dnowext 3dnow constant\_tsc \\
		                & rep\_good nonstop\_tsc extd\_apicid amd\_dcm pni monitor \\
		                & cx16 popcnt lahf\_lm cmp\_legacy svm extapic cr8\_legacy \\
		                & abm sse4a misalignsse 3dnowprefetch \\
		                & osvw ibs skinit wdt nodeid\_msr \\
		bogomips        & 3999.71 \\
		TLB size        & 1024 4K pages \\
		clflush size    & 64 \\
		cache\_alignment & 64 \\
		address sizes    & 48 bits physical, 48 bits virtual \\
		power management & ts ttp tm stc 100mhzsteps hwpstate \\
		\hline
	\end{tabular}
\end{table}


The system is equipped with 32 GB RAM.

The cache on this type of processors:

\begin{table}
	\begin{tabular}{|l|l|}
	\hline
		LEVEL1\_ICACHE\_SIZE                 & 65536 \\
		LEVEL1\_ICACHE\_ASSOC                & 2 \\
		LEVEL1\_ICACHE\_LINESIZE             & 64 \\
		LEVEL1\_DCACHE\_SIZE                 & 65536 \\
		LEVEL1\_DCACHE\_ASSOC                & 2 \\
		LEVEL1\_DCACHE\_LINESIZE             & 64 \\
		LEVEL2\_CACHE\_SIZE                  & 524288 \\
		LEVEL2\_CACHE\_ASSOC                 & 16 \\
		LEVEL2\_CACHE\_LINESIZE              & 64 \\
		LEVEL3\_CACHE\_SIZE                  & 10485760 \\
		LEVEL3\_CACHE\_ASSOC                 & 96 \\
		LEVEL3\_CACHE\_LINESIZE              & 64 \\
		LEVEL4\_CACHE\_SIZE                  & 0 \\
		LEVEL4\_CACHE\_ASSOC                 & 0 \\
		LEVEL4\_CACHE\_LINESIZE              & 0 \\
	\hline
	\end{tabular}
\end{table}


The assignment was aimed at measuring various performance parameters, namely L1 cache miss rate, L2 cache miss rate, Mflops/sec, runtime.

To reach our goals we have used special functions and counters from PAPI library.
Unfortunately, given system did not allow to measure more than 4 parameters at the same time, therefore, the source code was changed in a way to compile different versions of the grogram with special counters inside.

It was discovered that for some reason even 3 parameters could not be measures at once (for example L1 miss number, L1 total accesses, L2 miss number).
The functins PAPI\_start\_counters() returned an error code after trying of measure 3 parameters.

The compilation process was automatised with simple Python script which compiled the code with different compiler options and used all geometry data files.

For a computation of Mflops/sec and runtime the function PAPI\_flops() wase used.
\end{document}





